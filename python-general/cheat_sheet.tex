\title{A Quick Python Cheat Sheet}
%% \author{
%%   Luke Hodkinson \\
%%   Center for Astrophysics and Supercomputing \\
%%   Swinburne University of Technology \\
%%   Melbourne, Hawthorn 32000, \underline{Australia}
%% }
\date{\today}

\documentclass[10pt]{article}
\usepackage{color}
\usepackage[usenames,dvipsnames]{xcolor}
\usepackage{amsmath}
\usepackage{amsfonts}
\usepackage{listings}
%% \usepackage[scaled]{beramono}
%% \renewcommand*\familydefault{\ttdefault}
%% \usepackage[Tl]{fontenc}

\newcommand{\deriv}[2]{\ensuremath{\frac{\mathrm{d}#1}{\mathrm{d}#2}}}
\newcommand{\sderiv}[2]{\ensuremath{\frac{\mathrm{d}^2#1}{\mathrm{d}#2^2}}}

\lstset{
  language=Python,
  showstringspaces=false,
  formfeed=\newpage,
  tabsize=4,
  basicstyle=\small\ttfamily,
  commentstyle=\color{BrickRed}\itshape,
  keywordstyle=\color{blue},
  stringstyle=\color{OliveGreen},
  morekeywords={models, lambda, forms, dict, list, str, import, dir, help,
   zip, with, open}
}

\begin{document}
\maketitle

\section{Getting Help}
\begin{lstlisting}
  # Get help for a module, function, class, etc.
  help('math')
  help('str')

  # Get a brief listing of names in namespace.
  import math
  dir(math)
  dir(str)
\end{lstlisting}

\section{Strings}
\begin{lstlisting}
  # Creation.
  s = "Hello world!"
  s = 'Hello world!'
  s = 'Hello' + ' ' + 'world!'

  # Access.
  print s[0] # returns 'H'
  print s[1] # returns 'e'

  # Operations.
  s.split() # returns ['Hello', 'world!']
  s = ' '.join(['Hello\\n', 'world\\n']) # s becomes 'Hello\\nworld\\n'
  s.splitlines() # returns ['Hello', 'world']
  s.strip('.,') # removes periods and commas from string
\end{lstlisting}

\section{Lists}
\begin{lstlisting}
  # Creating.
  l = []
  l = [1, 'two', 3]
  l = list([1, 'two', 3]) # copy a list

  # Modifying lists.
  l.append(4)
  l.extend([5, 6, 7])
  l.reverse()
  l.sort()

  # Querying lists.
  4 in l
  l.count(1) # how many 1's in the list?

  # Iteration.
  for item in l:
    ...
\end{lstlisting}

\section{Dictionaries}
\begin{lstlisting}
  # Creating dictionaries.
  d = {}
  d = {1: 'one', 2: 'two'}
  d = dict([(1, 'one'), (2, 'two'), (3, 'three')])
  d = dict(zip([1, 2, 3], ['one', 'two', 'three']))

  # Modifying dictionaries.
  d[5] = 'five'
  d.pop(1) # removes key/value, returns value

  # Querying dictionaries.
  1 in d
  d.keys() # returns list of keys
  d.values() # returns list of values
  d.items() # returns list of key/value tuples.

  # Iteration.
  for key in d.iterkeys(): # iterate over keys
    ...
  for value in d.itervalues(): # iterate over values
    ...
  for key, value in d.iteritems(): # iterate over keys, values
    ...
\end{lstlisting}

\section{Files}
\begin{lstlisting}
  # Open a text file for reading.
  f = open('input.txt')

  # Open a text file for writing.
  f = open('output.txt', 'w')

  # Safely open a text file for reading.
  with open('input.txt') as f:
    # Do something here...
    # File is automatically closed when we
    # exit from the 'with' block.

  # Iterate over lines in a file.
  with open('input.txt') as f:
    for line in f:
      # Do something here...
\end{lstlisting}

\section{Functions}
\begin{lstlisting}
  # Declare a function.
  def func():
    # Do something here...

  # Accept arguments.
  def magnitude(x, y, z):
    mag = sqrt(x*x + y*y + z*z)

  # Return values.
  def magnitude(x, y, z):
    mag = sqrt(x*x + y*y + z*z)
    return mag

  # Calling functions.
  value = magnitude(3, 4, 5)
\end{lstlisting}

\section{List Comprehensions}
\begin{lstlisting}
  a_list = [1, 2, 3, 4]

  # Square each element of a list.
  a_list = [l*l for l in a_list]

  # Filter a list.
  a_list = [l for l in a_list if l < 3]

  # Filter and square each element in a list.
  a_list = [l*l for l in a_list if l < 3]
\end{lstlisting}

\section{Modules}
\begin{lstlisting}
  # Import a module and access contents.
  import math
  dir(math)
  math.cos(0.3)
  math.sin(0.3)

  # Import all module contents directly.
  from math import *
  cos(0.3)
  sin(0.3)

  # Import one symbol from module.
  from math import cos
  cos(0.3)
  sin(0.4) # error!
\end{lstlisting}

\section{Class Basics}
\begin{lstlisting}
  # Define an empty class.
  class MyClass:
    pass

  # Define a constructor.
  class MyClass:

    def __init__(self):
      self.some_int = 10
      self.my_name = 'MyClass'

  # Define methods.
  class MyClass:

    def __init__(self):
      self.some_int = 10

    def get_some_int_squared():
      return self.some_int*self.some_int
\end{lstlisting}

\end{document}
